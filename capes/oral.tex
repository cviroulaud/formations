\documentclass[a4paper,11pt]{article}
\input{/home/tof/Documents/Cozy/latex-include/preambule_doc.tex}
\input{/home/tof/Documents/Cozy/latex-include/preambule_commun.tex}
\newcommand{\showprof}{show them}  % comment this line if you don't want to see todo environment
\setlength{\fboxrule}{0.8pt}
\fancyhead[L]{\fbox{\Large{\textbf{CAPES}}}}
\fancyhead[C]{\textbf{éléments de langage}}
\newdate{madate}{10}{09}{2020}
%\fancyhead[R]{\displaydate{madate}} %\today
%\fancyhead[R]{Seconde - SNT}
%\fancyhead[R]{Première - NSI}
%\fancyhead[R]{Terminale - NSI}
\fancyfoot[L]{\vspace{1mm}Christophe Viroulaud}
\AtEndDocument{\label{lastpage}}
\fancyfoot[C]{\textbf{Page \thepage/\pageref{lastpage}}}
\fancyfoot[R]{\includegraphics[width=2cm,align=t]{/home/tof/Documents/Cozy/latex-include/cc.png}}

\begin{document}
Je suis enseignant depuis 20 ans dont 10 en lycée professionnel de banlieue parisienne et 10 en EREA. Je suis arrivé dans cette \guill{nouvelle carrière} avec une expérience et surtout une méthodologie particulière. En effet, en EREA devant ces élèves en grandes difficultés je ne réalisais quasiment plus (voire plus du tout) de cours magistral. Et ce pour plusieurs raisons :
\begin{itemize}
    \item contenu moins dense,
    \item mise en place de la co-intervention (association d'un enseignant de la partie professionnelle avec l'enseignant de mathématiques/français).
\end{itemize}
J'appliquais une approche plus \guill{déstructurée}:
\begin{itemize}
    \item Je posais une problématique du monde professionnel ou de la vie quotidienne.
    \item Je validais leur compréhension.
    \item Ils réalisaient des activités permettant de découvrir les nouvelles notions, utiles à la résolution de la problématique.
    \item Ces activités n'abordaient qu'une capacité/notion à la fois (découverte d'une formule, conversion, mesures, expérience\dots)
\end{itemize}
Il y a plusieurs avantages à cette approche:
\begin{itemize}
    \item pouvoir gérer l'hétérogénéité du groupe.
    \item créer une dynamique de réussite: même si tout le monde n'arrive pas à répondre à la problématique, tous ont le sentiment d'avoir réussi quelque chose.
    \item mettre les élèves en situation de pratique (et donc actifs de leur apprentissage) un maximum de temps effectif
    \item me permettre d'avoir une approche plus personnalisée (j'y reviendrais).
\end{itemize}
Ma problématique principale en ce début d'année (et donc de mon premier cours) était de savoir si je pouvais adapter ma méthodologie à ce nouveau profil d'élèves. En effet, même si j'ai obtenu des résultats avec un public d'EREA, la mise en place d'une telle pratique peut s'avérer chronophage. Et je n'avais pas, en ce début d'année, le recul nécessaire  sur le tempo de la progression jusqu'aux échéances pour les élèves de terminale (épreuves de spécialités en mars). Cependant il me semblait nécessaire d'utiliser mon expérience. De plus j'ai \guill{la chance} d'avoir des petits effectifs. Je savais donc que je pouvais accorder un temps individualisé important à chacun.\\
J'ai opté pour une approche hybride, à savoir :
\begin{itemize}
    \item Partir d'une problématique contextualisée. Ainsi, en terminale j'ai commencé ma progression en abordant la POO. Cependant ma séquence avait pour thème le jeu \emph{Fortnite} et une problématique posée ainsi : \guill{Quel concept de programmation mettre en place pour manipuler les combattants et les faire évoluer dans une partie ?}
    \item Faire ensuite réfléchir les élèves sur les solutions qu'ils pouvaient déjà mettre en place : utilisation de listes, dictionnaires pour les caractéristiques des personnages, création de fonctions pour leur évolution.
    \item Reprendre la main pour centraliser les approches possibles, en montrer les limites.
    \item Puis présenter ce nouveau paradigme (pour eux) qu'est la POO. Ici encore plusieurs étapes :
    \item \begin{itemize}
        \item Création d'une classe, du constructeur et des attributs.
        \item Mise en pratique en reprenant les contenus précédents.
        \item Création des méthodes pour modifier les caractéristiques du personnage.
        \item Encore une mise en pratique.
        \item Enfin instanciation et mise en pratique encore une fois.
    \end{itemize}
\end{itemize}
Même si le déroulé est plus linéaire que ce que je mettais en place en EREA, la différenciation est toujours bien présente : pour chaque activité, un socle minimal est demandé et des questions permettent d'aller plus loin. Les élèves les plus avancés vont au-delà des attendus minimaux (en produisant d'autres méthodes à la classe par exemple; ceci est d'autant plus possible que la contextualisation du cours leur permet de proposer des idées de développement) tandis que pour ceux plus en difficultés je décortique la construction de la classe avec eux.\\
Je donne ensuite une correction possible de l'ensemble des questions. Au niveau de la trace écrite, je donne aux élèves un document papier reprenant le contenu de la séquence (à savoir problématique, éléments de cours, activités) et je commente et déroule les séances à l'aide d'un diaporama qui reprend le fil du document et apporte les corrections. Ces documents sont ensuite accessibles sur un site web dédié pour une consultation ultérieure.\\
Cette approche où les élèves sont très rapidement et très souvent dans la pratique a une autre finalité : me permettre de repérer les points de difficultés \guill{atomiques}. Un exemple qui peut paraître dérisoire : en EREA quand un élève n'arrive pas à construire un simple tableau, il lui est difficile de se concentrer sur la notion de proportionnalité. Cette mise en pratique \guill{intensive} m'a très rapidement permis de repérer que certains élèves de terminales avaient des difficultés avec le typage des variables. C'est à dire qu'il arrive que certains essaient de récupérer un élément (de tableau) d'un entier. Plus généralement les élèves sont confus dans la manipulation des variables du problème. Qui est quoi? Quelles structures utilisons-nous? Cette difficulté est gênante pour le thème de la POO où la notation par point pour accéder aux attributs de la classe peut déstabiliser davantage encore. Pour tenter de surmonter cette difficulté j'ai imposé le typage explicite des paramètres des fonctions, des méthodes, fortement conseillé pour les variables (je vais d'ailleurs certainement l'imposer également).

intégration dans un "vrai jeu"?

J'ai essayé d'appliquer cette méthodologie à chaque séquence. Ainsi je n'ai pas fait un cours sur la POO mais j'ai présenté un nouvel outil pour construire un personnage de Fortnite et ses caractéristiques. Je n'ai pas fait un cours sur les graphes mais j'ai aidé à préparer la séquence de course d'orientation du professeur d'EPS. Je n'ai pas fait un cours généraliste sur les algorithmes \guill{diviser pour régner} mais j'ai regardé ce qui se cachait derrière la fonction native Python \emph{sorted} et donc quels tris étaient implémentés dans le timsort. Ce dernier exemple m'a permis de réutiliser le tri par insertion vu en première et donc plus généralement je ne me prive pas d'un retour sur des notions déjà abordées.\\
Je pense cette approche pertinente car elle donne un fil directeur et donc un sens à la séquence pour les élèves. D'autre part la mise en activité systématique des élèves me permet de pointer plus rapidement les lacunes. Je sais cependant que ces conclusions sont un peu biaisés du fait des petits effectifs de mes classes cette année et une adaptation sera certainement nécessaire l'année prochaine au vu de l'effectif qui s'annonce en première.

si temps: parler des interventions dans la vie du lycée:
\begin{itemize}
    \item club info
    \item organisation des JPO
    \item formations enseignants: Lycée connecté, JPO, Python
\end{itemize}
adaptations:
\begin{itemize}
    \item gestion du temps plus rigoureuse,
    \item certaines centralisations non systématiques, pas pour tous les élèves, via le slides uniquement pour certains qui avancent plus rapidement
\end{itemize}


d'autres points de difficultés?
réutilisation systématique des notions déjà abordées (exemple du timsort - révision tri par insertion-, paradigme fonctionnel -création d'un outil pour comparer les tris = révisions des tris-, POO et récursivité présentes partout)
\vspace{10em}
\begin{itemize}
    \item \textbf{cours inversé}, visio en petits salons
    \item 20 ans d'expérience dont 10 en lycée professionnel de banlieue parisienne et 10 ans en EREA.
    \item Je n'ai connu que des publics difficiles, en difficultés $\;\rightarrow\;$ nécessité d'adopter ma méthodologie pour pouvoir faire passer mon contenu.
    \item Je me suis tourné très tôt vers le numérique comme outil pour différencier mes apports.
    \item m'a permis d'apprendre différents langages mais également d'assimiler les différents concepts (POO, relation client-serveur - AJAX, \dots)
    \item apprentissage en autodidacte: création de sites, développement de logiciels (Flash), développement d'applications smartphone (pour non francophone par exemple).
    \item concours NSI m'a permis de donner un sens / objectif à toutes ces connaissances accumulées.
    \item construction des documents:
          \begin{itemize}
              \item doc + slides
              \item doc "épuré" $\;\rightarrow\;$ réflexion sur le contenu, éviter format "catalogue" (exemple de cours qui liste les méthodes des list Python\dots), permet prise de notes des élèves
              \item slides n'est pas qu'une répétition mais apporte autre chose: détails, décortiquage (exemple: chemin dans graphe), correction
              \item site regroupe doc et slides $\;\rightarrow\;$ éliminer papier à terme?
          \end{itemize}
    \item construction du cours: beaucoup de moment d'activité des élèves
    \item projection année prochaine:
    \begin{itemize}
        \item changement progression des 1ere: approche historique = partir du bit $\;\rightarrow\;$ Python (haut-niveau) en passant par langage machine \dots
        \item serveur NSI: approche "pragmatique" $\;\rightarrow\;$ démarrer sur \emph{système d'exploitation} avec mise en pratique très concrète des savoirs
    \end{itemize}
\end{itemize}

\begin{itemize}
    \item \textbf{problématique principal de début d'année:} enseignant depuis 20 ans, avec que des élèves en difficultés $\;\rightarrow\;$ comment adapter ma méthodologie au nouveau profil d'élèves
    \item avant peu voire plus du tout de cours magistral $\;\rightarrow\;$ contenu moins dense donc méthodologie plus facile à mettre en place (beaucoup de transversalité, maths = outils pour le monde professionnel); cours en co-intervention $\;\rightarrow\;$ très contextualisé
    \item approche (CAP) pour développer des compétences transversales (autonomie, prise de décision\dots); en lycée j'évalue ces compétences autrement (correction sur site, exercices en \guill{libre-service})
    \item comment adapter cette méthodologie?
    \item en CAP: cours très "déstructuré" = multiples activités indépendantes sur le même thème (à détailler) $\;\rightarrow\;$ pas encore mis en place avec mes élèves
    \item repérer les points de difficultés "atomiques": exemple = typage des données utilisées (encore en terminale, certains élèves essaient de récupérer un élément d'un entier) $\;\rightarrow\;$ si leur attention est focalisée sur ces bases, impossibilité d'assimiler le reste = problématique pour la POO. (équivalent en EREA: construire un tableau $\;\rightarrow\;$ difficile de comprendre la proprotionnalité). solution: typage des données systématique
    \item CAPPEI (certificat d'aptitude professionnelle aux pratiques de l'éducation inclusive)
    \item méthodo hybride: magistral / activité
    \item objectif (difficile à appliquer tout le temps): faire de l'activité "autonome" qui emmène vers une problématique globale (ex énergie électrique)
\end{itemize}
\end{document}